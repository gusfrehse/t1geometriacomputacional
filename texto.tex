% vim: set tw=80 ts=2 sts=2 sw=2 et:
\documentclass[12pt]{article}

\usepackage{mathtools}
\usepackage{amsfonts}
\usepackage{hyperref}

\title{Trabalho 1 de Geometria Computacional}
\author{Gustavo Silveira Frehse GRR20203927}
\begin{document}
  \maketitle
  \section{O Problema}
  Dados os vértices de $m$ polígonos, queremos descobrir se tratam-se de
  polígonos simples e se são convexos.

  Além disso, dado $n$ pontos queremos descobrir se estão dentro dos polígonos
  dados anteriormente.

  \section{Classificação Polígonos}
  \subsection{Teste Convexidade} Primeiro vamos observar que podemos ``criar'' a
  operação $\text{det} : V \times V \rightarrow \mathbb{R}$, tal que dada dois
  vetores $u$ e $v$ temos 
  \[
    \text{det}(u, v) = \text{determinante da matriz com colunas } u \text{ e } v
  \]
  Com essa operação podemos determinar a orientação de dois vetores.

  Para cada vértice $v_i$ checamos se $\text{det}(v_i - v_{i-1}, v_{i+1} - v_i)$
  tem o mesmo valor que para $v_{i - 1}$.

  Isso é equivalente a dizer que, caminhando em cima das arestas do polígono,
  nos vértices viramos sempre para à direita ou sempre à esquerda.

  Isso, por sua vez, é equivalente a dizer que o polígono é convexo. Assim, caso
  encontremos um $v_i$ em que $\text{det}(v_i - v_{i-1}, v_{i+1} - v_i)$ é
  diferente do anterior, sabemos que o polígono não é convexo.

  \subsection{Teste Simplicidade}
  Para cada par de arestas, checamos se elas se intersectam. Caso intersectem, o
  polígono não é simples.

  Para testar a interseção usamos a função \verb|intersect|
  \cite{wik-intersect}, que dada duas retas $a$ e $b$, cada uma especificada por
  dois vértices ($a_1, a_2, b_1, b_2$), calcula os parâmetros $t_0$ e $t_1$ tal
  que 

  \[
    a(t_0) = t_0 \cdot a_1 + (1-t_0) \cdot a_2 = t_1 \cdot b_1 + (1 - t_1) \cdot
    b_2 = b(t_1)
  \]

  Dessa forma conseguimos saber se duas retas se intersectam se os dois $t_0$ e
  $t_1$ estão entre 0 e 1.

  \section{Pontos interiores}
  Contamos o número de interseções com o polígono de uma semireta iniciando no
  ponto em questão. Se esse número é par, o ponto está fora, senão está dentro.

  Isso funciona pois sabemos que ao ``atravessar'' uma aresta, ou entramos no
  polígono ou saímos. Como se trata de uma semireta, sabemos que o último passo
  vai ser ``sair'' do polígono.

  Assim, caso o número de interseções seja par, tivemos que entrar e sair um
  número igual de vezes, e como terminamos fora, no início também estávamos
  fora. Caso seja ímpar, tivemos que sair uma vez a mais, portanto estávamos
  dentro do polígono.

  \begin{thebibliography}{9}
    \bibitem{wik-intersect} Wikipedia,
      \href{https://en.wikipedia.org/wiki/Line-line_intersection}{Line-line
      intersection}, 2023
  \end{thebibliography}


\end{document}
